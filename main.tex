\documentclass[11pt, a4paper]{jsarticle}
\usepackage{geometry}
\usepackage{cite}
\usepackage{url}
\usepackage[dvipdfmx]{graphicx}
\usepackage{amsmath}
\usepackage{float}
\usepackage{here}
\usepackage{listings,jvlisting} 
\lstset{
  basicstyle={\ttfamily},
  identifierstyle={\small},
  commentstyle={\smallitshape},
  keywordstyle={\small\bfseries},
  ndkeywordstyle={\small},
  stringstyle={\small\ttfamily},
  frame={tb},
  breaklines=true,
  columns=[l]{fullflexible},
  numbers=left,
  xrightmargin=0zw,
  xleftmargin=3zw,
  numberstyle={\scriptsize},
  stepnumber=1,
  numbersep=1zw,
  lineskip=-0.5ex
}

\geometry{left=25mm,right=25mm,top=25mm,bottom=30mm}

%ここまではライブラリ(?)とか文書の様式とか。よくわかってない。とりあえずコピペしとけ

\title{
ここにタイトルを入れる}
\author{
学修番号とか氏名とか入れようね
}
\renewcommand{\lstlistingname}{コード} %lstlistingのところで解説
\begin{document} %\begin{document}から\end{document}までに文章を書くこと
\date{提出日とか入れようね}
\maketitle
\section{section} %\section{}って書くと新しい段落がつくれる。勝手に段落番号振ってくれて便利だね
あ

\section{itemize}
\begin{itemize} %itemize.tex読んでくれ
\item[(1)]あ
\item[(2)] い
\end{itemize}

\section{subsection}
\subsection{subsection} %\subsection{}って書くと段落の中でさらに分けれる。勝手に段落番号振ってくれて便利だね

\section{table}
表\ref{table}に表を示す. %\ref{}って書くとついてるラベルの表やら式やら図やらを引用できるよ
\begin{table} [H] %table.tex読んでくれ
\caption{あ} 
\centering
\begin{tabular}{rr}\\ \hline
あ&い\\ \hline
う&え\\ 
お&か\\ \hline
\end{tabular}
\label{table}  
\end{table}

\section{figure}
図\ref{picture}に図を示す. %figure.tex読んでくれ
\begin{figure}[H]
\centering
\includegraphics[width=0.3\linewidth]{picture.jpg}
\caption{picture}
\label{picture}
\end{figure}

\section{equation} %equation.tex読んでくれ
式\ref{eq:1}に式を示す.
\begin{equation}
\label{eq:1}
ax^2 + bx + c = 0
\end{equation}

式\ref{eq:2}に式を示す.
\begin{align}
\label{eq:2}
a \left(x^2 + \frac{b}{a}x \right) + c &= 0\\
a \left(x^2 + \frac{b}{a}x + \frac{b^2}{4a^2} \right) - \frac{b^2}{4a} + c &= 0
\end{align}

\section{lstlisting}
コード\ref{lstlisting}にコードを示す
\begin{lstlisting}[caption=code.cpp,label=lstlisting]
#include <iostream>
using namesoace std;

int main(){
    cout << "Hello World" << endl;
    return 0;
}
\end{lstlisting}

\bibliographystyle{junsrt} 
\begin{thebibliography}{99}
    \bibitem{thebibliography} \LaTeX 概論
\end{thebibliography}

\end{document}