\begin{table} [H] 
\caption{}
\centering
\begin{tabular}{}\\ \hline

\end{tabular}
\label{}  
\end{table}

tableを使うと表の作成ができる。
下の例みたいに書くとmain.texの表1と同じものができる

\begin{table} [H] %表の書式とかを決めるよ
\caption{あ}  %表のタイトルを入力
\centering %中央揃え
\begin{tabular}{rr}\\ \hline %ここから表の中身書き始めるよ
あ&い\\ \hline 要素同士は&で区切る
う&え\\ 
お&か\\ \hline
\end{tabular}
\label{table} %表のラベル付けだよ、\ref{table}って書くとこの表を参照できるよ
\end{table}

こまかいもろもろ
・[H]は表をここに入れるぜっていう合図、順番がぐちゃぐちゃにならない [h][t]とかのオプションもある
・ \hlineって書くとその行の下に横線引けるよ、もっかい書くともう1本引けるよ
・18行目「rr」は「この表には要素が2個あってどっちも右揃えだよ」の意味、lは左詰めでcは中央揃えだから要素が4個で全部左揃えにしたかったらllllだね